\chapter{Results}
\begin{flushleft}
In conclusion we can say that we have developed a working OTA protocol.
You will have seen or will see in the demo that the OTA update, on the board, takes place without problems.
It is possible to repeat the operation several times and multiplexed on multiple devices.
\begin{flushleft}
\section{Known Issues}
\end{flushleft}
The major problem that we see on this solution is the server used.
From one point of view, Google drive provides us advantages in terms of simplicity of use and versatility; Drive already gives us a well-developed server that is easy to use and within everyone's reach and in addition, the Certified CA that is very important for the security side.

But using Google Drive, that is a third party server, accessible from anyone, does not give us absolute security during the loading and saving phase of the .bin.
For sure we can guarantee for the download and upload a HTTPS protocol, but while the file is on the server someone can stole my credential and modify my new software.

One solution could be create a own server with own CA, preferably on a local network, but this takes away the possibility of uploading the board remotely away from the device.
Another possibility could be adding a subsequent hash check that we try to explain in the Future Work Chapter.
\begin{flushleft}
\section{Future Work}
\end{flushleft}

\begin{flushleft}
\subsection{Cooperative hash check}
\end{flushleft}
At the end of the OTA update process the new software is safely downloaded from Google Drive. We want to add the possibility, perhaps under a specific command, to verify the hash of the .bin file downloaded by another IoT connected on the same network.
Below is represented one operations sequence that allows you to do this check.
\begin{itemize}
    \item Master send to Checker Validation Request
    \item Checker send to Server the Request download Software of master IoT
    \item Server Answer to checker with the download Software of master IoT
    \item Checker Saves in the partition that is not using (or P2 or P3) and calculate Hash of that partition
    \item Checker Request to Master the Hash of partition that is running
    \item Master Answer to Checker with the digest string
    \item Checker verify and validate the update
\end{itemize}
\paragraph{Precondition}

\begin{itemize}
    \item The IoT devices to perform this check need to be in the same network
    \item All the IoT devices are configurable as Master or Checker depending on the check request
    \item All the IoT devices have a static address. All the IoT devices know all the address of the cluster 
    \item The links to download the bin of other IoT device is hard-coded on all the board. This means that an IoT device can download all the software related to its cluster

\end{itemize}

\includegraphics[width=1\linewidth]{images/check hash.jpg}
\begin{flushleft}
\subsection{Generation hash for inter cluster check}
\end{flushleft}
\paragraph{Using FreeRTOS library}
\paragraph{}
\fcolorbox{black}{gray!30}{\url{https://www.freertos.org/pkcs11/pkcs11-mechanisms-digest-demo.html}}

Core PKCS 11 is a library belonging to FreeRTOS that allows us to hash a given array.
At the link above you will find an example of how the digest of the SHA256 function is calculated.
\begin{verbatim}
/***************************************************************************************/
/* Hash with SHA256 mechanism. */
xDigestMechanism.mechanism = CKM_SHA256;

/* Initializes the digest operation and sets what mechanism will be used
* for the digest. */
xResult = pxFunctionList->C_DigestInit( hSession,
                                        xDigestMechanism );
configASSERT( CKR_OK == xResult );

/* Pass a pointer to the buffer of bytes to be hashed, and its size. */
xResult = pxFunctionList->C_DigestUpdate( hSession,
                                          pxKownMessage,
                                          sizeof( pxKownMessage ) - 1 );
configASSERT( CKR_OK == xResult );

/* Retrieve the digest buffer. Since the mechanism is an SHA-256 algorithm,
* the size will always be 32 bytes. If the size cannot be known ahead of time,
* a NULL value to the second parameter xDigestResult, will set the third parameter,
* pulDigestLen to the number of required bytes. */
xResult = pxFunctionList->C_DigestFinal( hSession,
                                         xDigestResult,
                                         ulDigestLength );
configASSERT( CKR_OK == xResult );
/***************************************************************************************/
\end{verbatim}

\paragraph{Using ESP library}
\paragraph{}
In our case is not possible to consider the load for the digest as a file or a string but better consider it as a portion of memory, in particular in this case a partition.
\fcolorbox{black}{gray!30}{\url{https://www.esp32.com/viewtopic.php?t=16327}}
\begin{verbatim}
/***************************************************************************************/
 #include <sys/stat.h>
 #include "esp_err.h"
 #include "esp_log.h"
+#include "esp_partition.h"
 #include "esp_spiffs.h"
 
 static const char *TAG = "example";
@@ -96,4 +97,13 @@ void app_main(void)
     // All done, unmount partition and disable SPIFFS
     esp_vfs_spiffs_unregister(NULL);
     ESP_LOGI(TAG, "SPIFFS unmounted");
+
+    uint8_t hash[32] = {0};
+    const esp_partition_t *p = esp_partition_find_first(ESP_PARTITION_TYPE_DATA,
+                                                        ESP_PARTITION_SUBTYPE_DATA_SPIFFS, NULL);
+    ESP_ERROR_CHECK( esp_partition_get_sha256(p, hash) );
+    printf("SPIFFS partition hash ");
+    for (int i = 0; i < sizeof(hash); i++) {
+        printf("%02x ", hash[i]);
+    }
+    printf("\n");
 }
/***************************************************************************************/
\end{verbatim}
\begin{flushleft}
\subsection{Communication between IoT (ESP32)}
\end{flushleft}
To allow the ESP32 to start communication among them, we can use for our project the Protocol Communication (Protocomm).

\fcolorbox{black}{gray!30}{\url{https://docs.espressif.com/projects/esp-idf/en/latest/esp32/api-reference/provisioning/protocomm.html}}
At this link is possible to see the documentation and related example how to use protocomm API for a secure communication session.
\end{flushleft}
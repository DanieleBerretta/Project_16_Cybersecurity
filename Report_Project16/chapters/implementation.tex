\chapter{Implementation Details}
\begin{figure}[h!]
\includegraphics[width=1\textwidth]{images/Generale.jpg}
\caption{Process Diagram}
\label{fig:generalschema} % This is the image label, with which you can refer to the image in any document location.
\end{figure}

In this section we can go on the implementation detail of this protocol, and the operational steps to set up the machine and work with ESP32 board.
As a remind, this is the folder project:
\fcolorbox{black}{gray!30}{\url{https://drive.google.com/drive/folders/12X8qwaMzzthq3UGNoWMIBG-sxKIKymrK}}
\section{Include}
\begin{flushleft}
\subsection{Wireless Connectivity}
\end{flushleft}
Wireless connectivity is required for the OTA update procedure. One of the key point for the procedure, is the usage of internet connectivity.

This could gives us potential benefits: we can run the update without connection to a local network, the devices can communicate even if they are not within the same intranet, but on the same time we could be more sensitive to cyber attacks.


As mentioned, we use an ESP32 board, that is equipped with Wi-Fi module and good library well explained to use it.
It also has good libraries for wireless communication and secure boot.

\begin{itemize}
    \item \fcolorbox{black}{gray!30}{esp\textunderscore wifi.h} - wifi connectivity
    \item \fcolorbox{black}{gray!30}{nvs\textunderscore flash.h} - wifi configuration
    \item \fcolorbox{black}{gray!30}{esp\textunderscore http\textunderscore client.h } - using client server 

\end{itemize}

\subsection{FreeRTOS}
FreeRTOS is used as an operating system for tasks and for interrupt handling.
In particular, an interrupt is used on one of the buttons on the board, to start the OTA download and control routine.
We could implement any other method to activate the procedure, eg. a message sent from another ECU, timer or any other signal, we had many GPIOs available, however the button seemed to us for the demonstration demo to be a quick and convenient solution to use this feature.
For future developments it is therefore possible to think about different solutions more suitable for the application of use.
Libraries used:
\begin{itemize}
    \item \fcolorbox{black}{gray!30}{freertos/FreeRTOS.h} - freertos library 
    \item \fcolorbox{black}{gray!30}{freertos/task.h} - freertos library
    \item \fcolorbox{black}{gray!30}{freertos/semphr.h} - freertos library
    \item \fcolorbox{black}{gray!30}{driver/gpio.h} - resource library
\end{itemize}

\begin{flushleft}
\section{Memory Partition}
\end{flushleft}
The memory of the board is divided in 3 partition of 1MB.
\begin{itemize}
    \item P0 : Partition flashed with the first Software, used also has back-up (Flashed with wired connection)
    \item P1-P2 : Partitions free for future update (Flash with OTA protocol)
\end{itemize}
\centering

\includegraphics[width=0.85\textwidth]{images/all.png}

\begin{flushleft}
\section {Start Communication}
\end{flushleft}

\fcolorbox{black}{gray!30}{\url{https://docs.espressif.com/projects/esp-idf/en/latest/esp32/api-reference/protocols/esp_http_client.html}}


\paragraph{}
\begin{flushleft}
\textbf{Communication happens using http client server. Configuration:}
\end{flushleft}

\begin{verbatim}
    // as we use http protocol:
    esp_http_client_config_t clientConfig = {
    .url = "Update Link", // here where we have the binary file
    .event_handler = client_event_handler,  
    .cert_pem = (char *)server_cert_pem_start // Certificate Cfg
\end{verbatim}

\begin{flushleft}
\section {Download and install in partition}
\end{flushleft}

\fcolorbox{black}{gray!30}{\url{https://docs.espressif.com/projects/esp-idf/en/latest/esp32/api-reference/system/esp_https_ota.html}}
\paragraph{}
OTA APIs used: 
\fcolorbox{black}{gray!30}{\texttt{esp\_https\_ota}}
\begin{verbatim}
    // esp_https_ota returns ok and the binary file is putted in the ota partition that we setted
    if(esp_https_ota(&clientConfig) == ESP_OK)
    {
      ESP_LOGI(TAG,"firmware update is successful now we are in version %d.",firmware_version);
      vTaskDelay(pdMS_TO_TICKS(5000)); //some delay before restarting 
      esp_restart(); //restaring our microcontroller and start from the new memory partition
    }
\end{verbatim}

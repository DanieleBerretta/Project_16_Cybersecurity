\chapter*{Abstract}

This document reports the development of the secure OTA update project, for the course Cybersecurity for Embedded Systems.
It describes how this OTA protocol was develop and then designed with a look at possible and future developments, starting from a simpler working base.
By Reading this report, we will review some concepts for wireless update (using an internet connection) and safety concepts that we want to guarantee and apply to our protocol.
The main actors in this protocol are a common PC, used to upload the new binary file to be downloaded to the device,
the internet connection to the exchange server (in this case we choose Google Drive as a prototype server to check that our mechanism works correctly),
and the boards used for IoT communications (ESP32s).

At the end of the report you will be able to implement step by step a working update protocol for devices that use ESP32, being aware of the secure mechanism used.
At the links below you can find the presentation slides with video demos and the source code of the project:

\begin{itemize}
\item Code: \fcolorbox{black}{gray!30}{\url{https://drive.google.com/drive/folders/12X8qwaMzzthq3UGNoWMIBG-sxKIKymrK}}
\item Presentation: \fcolorbox{black}{gray!30}{\url{https://1drv.ms/p/s!AvgcmezmQyo_k142UmFzocKHarcV?e=8MCfQo}}
\end{itemize}
